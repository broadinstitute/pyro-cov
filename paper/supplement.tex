% Use only LaTeX2e, calling the article.cls class and 12-point type.

\documentclass[12pt]{article}

% Users of the {thebibliography} environment or BibTeX should use the
% scicite.sty package, downloadable from *Science* at
% http://www.sciencemag.org/authors/preparing-manuscripts-using-latex 
% This package should properly format in-text
% reference calls and reference-list numbers.

\usepackage{scicite}
\usepackage{times}

% Custom packages.
\usepackage{amsmath}
\usepackage{hyperref}

% The preamble here sets up a lot of new/revised commands and
% environments.  It's annoying, but please do *not* try to strip these
% out into a separate .sty file (which could lead to the loss of some
% information when we convert the file to other formats).  Instead, keep
% them in the preamble of your main LaTeX source file.


% The following parameters seem to provide a reasonable page setup.

\topmargin 0.0cm
\oddsidemargin 0.2cm
\textwidth 16cm 
\textheight 21cm
\footskip 1.0cm


%The next command sets up an environment for the abstract to your paper.

\newenvironment{sciabstract}{%
\begin{quote} \bf}
{\end{quote}}



% Include your paper's title here

\title{Supplementary material} 


% Place the author information here.  Please hand-code the contact
% information and notecalls; do *not* use \footnote commands.  Let the
% author contact information appear immediately below the author names
% as shown.  We would also prefer that you don't change the type-size
% settings shown here.

\author
{
% John Smith,$^{1\ast}$ Jane Doe,$^{1}$ Joe Scientist$^{2}$\\
% \\
% \normalsize{$^{1}$Department of Chemistry, University of Wherever,}\\
% \normalsize{An Unknown Address, Wherever, ST 00000, USA}\\
% \normalsize{$^{2}$Another Unknown Address, Palookaville, ST 99999, USA}\\
% \\
% \normalsize{$^\ast$To whom correspondence should be addressed; E-mail:  jsmith@wherever.edu.}
}

% Include the date command, but leave its argument blank.

\date{}



%%%%%%%%%%%%%%%%% END OF PREAMBLE %%%%%%%%%%%%%%%%



\begin{document} 

% Double-space the manuscript.

\baselineskip24pt

% Make the title.

\maketitle 



% Place your abstract within the special {sciabstract} environment.

% \begin{sciabstract}
%   This document presents a number of hints about how to set up your
%   {\it Science\/} paper in \LaTeX\ .  We provide a template file,
%   \texttt{scifile.tex}, that you can use to set up the \LaTeX\ source
%   for your article.  An example of the style is the special
%   \texttt{\{sciabstract\}} environment used to set up the abstract you
%   see here.
% \end{sciabstract}



% In setting up this template for *Science* papers, we've used both
% the \section* command and the \paragraph* command for topical
% divisions.  Which you use will of course depend on the type of paper
% you're writing.  Review Articles tend to have displayed headings, for
% which \section* is more appropriate; Research Articles, when they have
% formal topical divisions at all, tend to signal them with bold text
% that runs into the paragraph, for which \paragraph* is the right
% choice.  Either way, use the asterisk (*) modifier, as shown, to
% suppress numbering.

\section*{Materials and methods}

\paragraph*{Data and Code Availability}

Source code for data preprocessing and modeling and available at
\url{https://github.com/broadinstitute/pyro-cov}.
GISAID sequence data is publicly available at
\url{https://gisaid.org}.
PANGO lineage aliases are available at \url{https://cov-lineages.org/} with source code at \url{https://github.com/cov-lineages/lineages-website} and lineage aliases available at \url{https://github.com/cov-lineages/pango-designation}.

\paragraph*{Data Preparation}

We used samples from GISAID (12) including labels for time, location, PANGO lineage annotation (7), and genetic sequence.
We discarded records with missing time, location, or lineage.
We called mutations using the nextclade tool \cite{aksamentov2020nextclade}.
Because PANGO lineages are genetically heterogeneous (with small variation within each strain), we created continuous $[0, 1]$-valued features denoting, for each (strain, mutation) pair, the portion of samples in that strain exhibiting the mutation.
We discarded mutations that did not occur in the majority of samples in any single strain; 2337 mutations passed this threshold.
We binned time intervals into 14-day segments, choosing a multiple of 7 to minimize weekly seasonality.
Because sample counts vary widely across GISAID geographic region, we aggregated regions into the following coarse partitions: first-level subregions of any country with a subregion with at least 5000 samples, and entire countries all of whose subregions had <5000 samples
We dropped regions without samples in at least two time intervals.

\paragraph*{Probabilistic Model}

% Comment by Sagar Gosai:
% Another way to help justify your choices of priors and conditional
% likelihoods would be to generate prior predictive samples for the final
% counts and see how well those match the marginal distribution of the data.
% 
% This paper by Gabry et. al. provides some nice front-end strategies to
% support your modeling choices:
% https://rss.onlinelibrary.wiley.com/doi/full/10.1111/rssa.12378.

We modeled relative strain growth with a hierarchical Bayesian regression model with multinomial likelihood.
Arrays in the model may index over times $t\in \{1,\dots,42\}$, PANGOlineages (strains) $s\in\{1,\dots,1281\}$, regions $r\in\{1,...,1070\}$, and amino acid mutations (features) $f\in\{1,\dots,2337\}$.
The model, shown below, regresses strain counts $y_{trs}$ in each time-region-strain bin against amino acid mutation covariates $X_{sf}$.
As this is a Bayesian regression model, variables $y$ and $X$ are observed and all other variables in the model are latent.
The portion of strains in a single time-region bin is modeled as a Multinomial distribution of a multivariate logistic growth function $\operatorname{softmax}(\alpha_{rs} + t\beta_{r\cdot})$ with intercept $\alpha_{rs}$ and slope $\beta_{rs}$.
The intercepts $\alpha_{rs}$ denote relative log prevalence of strain $s$ in region $r$; these are modeled hierarchically around a global relative log prevalence $\alpha_s$ of each strain.
The slopes $\beta_{rs}$ are modeled hierarchically around a per-strain growth rate $\sum_i \beta_f X_{sf}$ that is linearly regressed against amino acid mutation features $X$.
We put weak priors on scale parameters $\sigma_1$, $\sigma_2$, and $\sigma_4$.
We fix the linear regression scale parameter $\sigma_3$ to a small value (relative to the estimated generation time $\tau=5.5$ days), forcing the regression problem towards a sparse solution (meaning most observed mutations have little effect on growth rate).
This small value was chosen to based on 2-fold cross validation.

\begin{align*}
  \textstyle
  \alpha_s &\sim \operatorname{Normal}(0, \sigma_1) &
  \sigma_1 &\sim \operatorname{LogNormal}(0, 2) \\
  \alpha_{rs} &\sim \operatorname{Normal}(\alpha_s, \sigma_2) &
  \sigma_2 &\sim \operatorname{LogNormal}(0, 2) \\
  \beta_f &\sim \operatorname{Logistic}(0,\, \sigma_3) &
  \sigma_3 &= \frac{1}{200\, \tau} \\
  \beta_{rs} &\sim \operatorname{Normal}\Bigl(
   \sum_i \beta_f X_{sf},\, \sigma_4
  \Bigr) &
  \sigma_4 &\sim \operatorname{LogNormal}(-4, 2) \\
  \underline{y_{trs}} &\sim \operatorname{Multinomial}\Bigl(
    \sum_s y_{trs},\, \operatorname{softmax}(\alpha_{rs} + t \beta_{r\cdot})_s
  \Bigr)
\end{align*}

\paragraph*{Probabilistic Inference}

The model was implemented in the Pyro probabilistic programming language \cite{bingham2019pyro}.

\bibliography{main}

\bibliographystyle{Science}


\end{document}




















